%*******************************************************
% Abstract
%*******************************************************
%\renewcommand{\abstractname}{Abstract}
\pdfbookmark[1]{Abstract}{Abstract}
\begingroup
\let\clearpage\relax
\let\cleardoublepage\relax
\let\cleardoublepage\relax

\chapter*{Abstract}
Neuronal networks must produce stable circuit output for sustained periods of time despite environmental perturbation. In addition, they must be sensitive to key endogenous signaling to produce differing output. The \acs{STG} manages these competing objectives while remaining degenerate to ion channel density. Neuromodulators can produce a diverse set of network states using the same cellular and synaptic morphology. In particular to the \acs{STG}, the dense, tangled neuropil and gradations in reversal potential render neurons isopotential with respect to the somata. Neuromodulators, then, play the role of maintaining and switching network activity. For stable and responsive biological activity, degenerate networks must still be robust to environmental perturbation and responsive to intentional modulation. In this thesis, I describe red pigment-concentrating hormone (\acs{RPCH}) acting as a neuromodulator on a computational model of a rhythmic motor circuit.

\vfill

\endgroup

\vfill
